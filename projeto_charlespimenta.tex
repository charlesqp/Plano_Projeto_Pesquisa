\documentclass[11pt,a4paper]{article}

\usepackage[brazil]{babel}
\usepackage[utf8]{inputenc}
\usepackage{ae}
\usepackage{indentfirst}
\usepackage[margin=2cm]{geometry}
\usepackage{amsthm,amssymb,amsfonts,amsmath}
\usepackage{graphicx}
\usepackage{psfrag}
\usepackage{booktabs}
\usepackage[numbers]{natbib}
\usepackage{enumitem}
\usepackage{colortbl,booktabs}

\bibliography{references}

\begin{document}

\thispagestyle{empty}

\noindent
\begin{minipage}{0.8\linewidth}
  {\Large\bf Universidade Federal de Minas Gerais}\\
  {\small ESCOLA DE ENGENHARIA}\\
  {\sc Departamento de Engenharia Elétrica}
\end{minipage} 
\hfill 
\begin{minipage}{3cm}
  \includegraphics[width=3cm]{ufmg_ext.pdf}
\end{minipage}

\vspace{1mm}

\noindent
\hrule

\vspace{2.0cm}

\vfill

\begin{center}
  \Large PROJETO DE PESQUISA
\end{center}

\vfill

\begin{center}
  \Large\textbf{MESTRADO}
\end{center}

\vfill

\begin{center}
  \LARGE \bf Projeto de observadores para estimação de estado em bancos de baterias de Lítio
\end{center}

\vfill

\begin{flushright}
\begin{minipage}{12.0cm}
{\bf Aluno:} Charles Quirino Pimenta \\
{\bf Orientador:}~Prof. Dr. Luciano Antonio Frezzatto Santos \\
\end{minipage}
\end{flushright}

\vfill

\begin{center}
  \today
\end{center}

\vfill

\noindent
\rule{6cm}{0.25mm} \hspace{4cm} \rule{6cm}{0.25mm}\\

\noindent
\hspace{0.50cm} Charles Quirino Pimenta \hspace{5.5cm} Luciano Antonio Frezzatto Santos

\newpage

\section{Introdução}

Impulsionado pelas mudanças climáticas, a busca pela redução de emissões de gases do 
efeito estufa veem impulsionando a busca por combustíveis alternativos e a 
comercialização de veículos elétricos (Evs) e híbridos (HEVs).

Nos últimos anos, sistemas dinâmicos associados a diferentes problemas
nas diversas áreas de engenharia têm sido estudados por meio de
modelos matemáticos cada vez mais complexos e completos. A existência
de distúrbios externos, incertezas paramétricas, dinâmicas não
modeladas, acoplamentos não lineares e, também, falhas (em atuadores, em
sensores, escorregamento, entre outras) são fatores que impactam o
funcionamento da planta, comprometendo seu desempenho ou, em casos
extremos, ocasionando instabilidades~\cite{XG:00,GC:14,LYCC:16}.


Em face dos problemas supracitados, desenvolver métodos que minimizem
ou anulem distúrbios externos e que sejam capazes de contornar
possíveis falhas no sistema, isto é, sejam tolerantes às falhas, tem
sido uma área de grande interesse em pesquisas recentes. Os principais
métodos existentes na literatura baseiam-se na filtragem de
Kalman~\cite{YCY:05,CH:16} e na síntese de 
observadores~\cite{LM:90,FF:12,WJ:13,ZDZ:16,CYGL:16} para a estimação
das falhas e distúrbios sobre a planta de interesse. Com respeito ao
projeto de observadores, os observadores com entradas desconhecidas
(em inglês, \emph{unknown input observers} --- UIOs) são largamente
utilizados tanto para a rejeição de distúrbios externos e correta
estimação dos estados do sistema quanto para a identificação de falhas
em diversos
contextos~\cite{TP:99a,Koe:05,Koe:06,LGBdS:11,IM:15,CYGL:16}.

Para realizar a síntese de UIOs pode-se, por exemplo, transformar o
problema em um procedimento de otimização convexa o qual pode ser
resolvido, em particular, por meio de programação semi-definida,
utilizando desigualdades matriciais lineares (em inglês, \emph{Linear
  Matrix Inequalities} --- LMIs) como
restrições~\cite{BEFB:94,EN:00,B-TEN:00}.

É inegável a importância dos resultados derivados da análise de
estabilidade por meio de funções de Lyapunov~\cite{Lya:92}. Ao longo
dos últimos anos, inúmeros trabalhos utilizando LMIs e funções de
Lyapunov proporcionaram estratégias para análise de estabilidade de
sistemas com incertezas, com extensões para a síntese de
controladores, filtros e observadores robustos. Os resultados para
análise de estabilidade e suas extensões evoluíram da função
quadrática com uma matriz de Lyapunov constante~\cite{BPG:89,GPB:91},
passando por formas afins, como em~\cite{dOBG:99,LP:03,PABB:00}
(utilizando variáveis extras advindas do Lema de
Finsler~\cite{dOS:01}) e~\cite{RP:01a,RP:02} (explorando diretamente a
expressão polinomial da LMI dependente de parâmetros), até chegar à
caracterização completa por meio de funções de Lyapunov polinomiais de
grau genérico construídas a partir de relaxações LMIs
convergentes~\cite{Bli:04a,CGTV:05b,Sch:05,OP:07a,SH:06,OdOP:08,CGTV:09,Che:10}.


Os objetivos principais deste plano de trabalho incluem o projeto de
UIOs e a síntese de controladores tolerantes a falhas para sistemas
dinâmicos (contínuos e discretos) utilizando funções de Lyapunov
dependentes de parâmetros, desenvolvendo, sempre que possível, as
condições sob a forma de LMIs. Os métodos desenvolvidos serão
validados por meio de simulações computacionais e, posteriormente,
serão realizados ensaios práticos para corroborar os resultados
obtidos.

A seguir é apresentada uma breve descrição do problema de projeto de
UIOs para sistemas dinâmicos contínuos no tempo.


\section{Descrição do problema}

Uma classe de sistemas lineares de particular interesse é a classe de
sistemas com parâmetros variantes no tempo, descritos genericamente
por
\begin{equation}
  \label{eq:slpv}
  \begin{aligned}
    \dot{x}(t) &= A(\xi(t)) x(t) + B(\xi(t)) u(t) + D(\xi(t)) d(t) +
    F(\xi(t)) f(t) \\
    y(t) &= C x(t)
  \end{aligned}
\end{equation}
sendo $x(t) \in \mathbb{R}^n$ o estado, $u(t) \in \mathbb{R}^m$ uma
entrada conhecida, $d(t) \in \mathbb{R}^d$ uma entrada de distúrbio
desconhecida, $f(t) \in \mathbb{R}^r$ um vetor de falhas do sistema e
$y(t) \in \mathbb{R}^p$ a saída. As matrizes
$(A(\xi(t)),B(\xi(t)),D(\xi(t)),F(\xi(t))) \in \mathcal{D}$, com
\begin{gather}
  \mathcal{D} = \left\{(A(\xi(t)),B(\xi(t)),D(\xi(t)),F(\xi(t))) =
    \sum_{i=1}^{N}\xi_i(t)(A,B,D,F)_i, \quad \xi_i(t) \geq 0, \quad
    \sum_{i=1}^{N}\xi_i(t) = 1\right\}
\end{gather}
ou seja, o conjunto $\mathcal{D}$ é um politopo descrito pela
combinação convexa dos vértices $(A,B,D,F)_i$, $i = 1, \dotsc, N$,
supostos conhecidos. Modelos assim podem ser também utilizados para
tratar sistemas nebulosos~\cite{TS:85,TW:01} ou especializados para o
caso de sistemas com parâmetros invariantes no tempo, isto é, sistemas
incertos.

Sem a presença de falhas e de distúrbios desconhecidos no sistema, um
observador de Luenberger~\cite{Lue:71} pode ser empregado para estimar
os estados não acessíveis do sistema a partir da saída
medida~$y(t)$. Todavia, a existência de tais sinais indesejáveis torna
a estrutura do observador de Luenberger não mais adequada para uma boa
estimação dos estados, fazendo-se, portanto, necessária a
implementação de um observador capaz de rejeitar os distúrbios
desconhecidos~$d(t)$ e estimar adequadamente tanto os estados do
sistema quanto as falhas~$f(t)$. Nesse sentido, surgem os UIOs como
uma alternativa para se estimar os estados e possíveis falhas do
sistema, ao mesmo tempo que os distúrbios são desconsiderados. De
forma geral, a estrutura de um UIO é dada por
\begin{equation}
  \label{eq:observer}
  \begin{aligned}
    \dot{z}(t) &= N(\xi(t)) z(t) + G(\xi(t)) u(t) + L(\xi(t)) y(t) \\
    \hat{x}(t) &= z(t) - H y(t)
  \end{aligned}
\end{equation}
sendo $z(t) \in \mathbb{R}^n$ o estado do observador e $\hat{x}(t) \in
\mathbb{R}^n$ o estado estimado do sistema.

Restrições adicionais como especificações de desempenho em termos de
normas $\mathcal{H}_2$ e $\mathcal{H}_\infty$ e presença de não
linearidades específicas, como saturação, podem também ser
consideradas. O projeto de controladores tolerantes a falhas baseados
em UIOs~\cite{BFKPS:00,BSW:01,YCY:05,CH:16} é outro tópico de bastante
interesse.

\subsection{Tópicos de investigação}

É objetivo deste plano adotar metodologias baseadas em programação
semi-definida para tratar problemas como os descritos acima, com
ênfase para novas contribuições para a síntese de UIOs e controladores
tolerantes a falhas. Nesse contexto, serão utilizadas funções de
Lyapunov polinomiais e LMIs. Entre outros, os seguinte problemas serão
investigados neste plano de trabalho:
\begin{itemize}
\item Projeto de observadores com entradas desconhecidas para sistemas
  incertos e variantes no tempo, podendo apresentar não linearidades;

\item Especificações de desempenho;
  
\item Projeto de controladores tolerantes a falhas;
\end{itemize}
Serão tratados modelos contínuos e discretos no tempo.

\section{Metodologia}

A metodologia básica utiliza programação semi-definida, funções de
Lyapunov polinomiais e condições escritas na forma de LMIs, que podem
ser solucionadas em tempo polinomial por pacotes
especializados~\cite{BEFB:94,EN:00,Stu:99,Lof:04,Mos:15}.

\section{Plano de trabalho e cronograma}

Este plano de mestrado está previsto para dezoito meses de duração,
compreendendo as seguintes etapas:

\begin{enumerate}[label=\textbf{\alph*)}]
\item Cumprimento dos créditos necessários para o programa de
  mestrado;
  
\item Levantamento bibliográfico, estudo de abordagens e ferramentas
  existentes, familiarização com os resultados baseados em programação
  semi-definida, funções de Lyapunov polinomiais e LMIs;
  
\item Extensão das abordagens existentes para o projeto de
  observadores com entradas desconhecidas considerando incertezas na
  matriz de saída, falhas e ruídos em medidas e desenvolvimento de
  novos métodos;
  
\item Simulações e implementações dos métodos desenvolvidos;
  
\item Ensaios práticos em plantas disponíveis na Escola de Engenharia;

\item Projeto de controladores tolerantes a falhas;
  
\item Redação de artigos técnicos;

\item Redação do manuscrito.
\end{enumerate}

O cronograma da Tabela~\ref{tab:cronograma} apresenta,
aproximadamente, a distribuição das etapas descritas ao longo dos
trimestres.

\begin{table}[!ht]
  \caption{Cronograma proposto para o projeto de mestrado.}
  \label{tab:cronograma}
  \begin{center}
    \begin{tabular}{ccccccc}
      & $1^{\tiny o}$ trim & $2^{\tiny o}$ trim & $3^{\tiny o}$ trim & $4^{\tiny o}$ trim & $5^{\tiny o}$ trim & $6^{\tiny o}$ trim \\
      \toprule
      \textbf{a)} & \cellcolor[rgb]{0.5,0.5,0.5}{} & \cellcolor[rgb]{0.5,0.5,0.5}{} & & & & \\
      \midrule
      \textbf{b)} & \cellcolor[rgb]{0.5,0.5,0.5}{} & \cellcolor[rgb]{0.5,0.5,0.5}{} & \cellcolor[rgb]{0.5,0.5,0.5}{} & & & \\
      \midrule
      \textbf{c)} & & \cellcolor[rgb]{0.5,0.5,0.5}{} & \cellcolor[rgb]{0.5,0.5,0.5}{} & \cellcolor[rgb]{0.5,0.5,0.5}{} & & \\
      \midrule
      \textbf{d)} & & \cellcolor[rgb]{0.5,0.5,0.5}{} & \cellcolor[rgb]{0.5,0.5,0.5}{} & \cellcolor[rgb]{0.5,0.5,0.5}{} & \cellcolor[rgb]{0.5,0.5,0.5}{} & \\
      \midrule
      \textbf{e)} & & & \cellcolor[rgb]{0.5,0.5,0.5}{} & \cellcolor[rgb]{0.5,0.5,0.5}{} & \cellcolor[rgb]{0.5,0.5,0.5}{} & \cellcolor[rgb]{0.5,0.5,0.5}{} \\
      \midrule
      \textbf{f)} & & & & & \cellcolor[rgb]{0.5,0.5,0.5}{} & \cellcolor[rgb]{0.5,0.5,0.5}{} \\
      \midrule
      \textbf{g)} & & & \cellcolor[rgb]{0.5,0.5,0.5}{} & \cellcolor[rgb]{0.5,0.5,0.5}{} & \cellcolor[rgb]{0.5,0.5,0.5}{} & \cellcolor[rgb]{0.5,0.5,0.5}{} \\
      \midrule
      \textbf{h)} & & & & & & \cellcolor[rgb]{0.5,0.5,0.5}{} \\
      \bottomrule
    \end{tabular}
  \end{center}
\end{table}

\bibliographystyle{plainnat}
\bibliography{listacompleta,sdp,filtering,polinomial,sispot,estab,fldp,hoofilt,robust,fuzzy,estgeral,bibluciano}

\end{document}
